	\documentclass{article}
	\usepackage{amsmath}
	\usepackage{nccmath}
	\DeclareMathSizes{10}{10}{10}{10}
	\setlength{\parindent}{0pt}
	\title{Konfluenz}
	\date{ }
	\begin{document}
	\maketitle
	%----------------------------------%
	\textbf{Referenzaufgabe:} \\
		\\
		\textbf{data List} a = $Nil$ $|$ $Cons$ a \textbf{List} a
		\begin{align*}
			snoc\;Nil\;a 			&= Cons\;a\;Nil\\
			snoc(Cons\;x\;xs)\;a	&= Cons\;x\;(snoc\;xs\;a)
		\end{align*}
		\begin{align*}
			Nil + ys 			&= ys\\
			(Cons\;x\;xs) + ys 	&= Cons\;x\;(xs + ys)
		\end{align*}
		\underline{Beweisen sie dass:}\\
		$\forall e,\;xs,\;ys\\xs+(Cons\;e\;ys) = (snoc\;xs\;e)+ys$ \\\\
	%----------------------------------%
	\textbf{Induktionsanfang:}\\\\
		$xs = Nil$\;\;$\Rightarrow$ Einsetzen und beide Seiten maximal vereinfachen
	\begin{align*}
		Nil+(Cons\;e\;ys) 	&= (snoc\;Nil\;e)+ys\\
			 Cons\;e\;ys	&= (snoc\;Nil\;e)+ys\\
			 Cons\;e\;ys	&= (Cons e Nil) + ys\\
			 Cons\;e\;ys	&= Cons\;e (Nil + ys)\\
			 Cons\;e\;ys	&= Cons\;e\;ys
	\end{align*}
		
	\begin{tiny}
	\copyright\ Joint-Troll-Expert-Group (JTEG) 2015
	\end{tiny}
\end{document}