%RK Klausur Loesungsversuch (faui2k13)
%compile with: latexmk -pdf $Dateinname
\documentclass{article}
\usepackage{amsmath}
\usepackage{hyperref}
\usepackage{tikz}
\usepackage{listings}
\DeclareMathSizes{10}{10}{10}{10}
\setlength{\parindent}{0pt}
\title{Klausur September 2014 - L\"osungsvorschlag"}
\author{L\"o"sungen: faui2k13, Latexversion: Sheppy}
\begin{document}
	\maketitle
	\section{Allgemeine Fragen}
			\textbf{a)}\\
				- Ausbreitungsverz\"ogerung\\
				- Warteschlangenverz\"ogerung\\
				- Bearbeitungsverz\"ogerung\\
				- \"Ubertragungsverz\"ogerung\\\\
			\textbf{b)}\\
				- Warteschlange voll\\
				- Physikalische probleme\\\\
			\textbf{c)}\\
				Damit soll verhindert werden, dass eine Station die \"Ubertragung eines kurzen Rahmens 
				beendet, bevor sein erstes Bit \"uberhaupt das andere Ende des Kabels erreicht,wo er dann 
				vielleicht mit einem anderen Frame kollidiert. \\\\
			\textbf{d)}\\
				- Wegen m\"oglichem ACK – Verlust /ACK – Verz\"ogerung/alternativ: Damit immer das ACK für 
				das letzte erfolgreiche Paket geschickt werden kann\\\\
			\textbf{e)}\\
				- Header mit fester L\"ange (für schnelles Weiterleiten)\\
				- keine Fragmentierung (wird einfach verworfen falls $>$ MTU)\\
				- keine Pr\"ufsumme (Fehlererkennung in h\"oheren Schichten)\\
				- zusätzliche Optionen außerhalb als nächster Header\\
				- zustandslose Autokonfiguration\\\\
			\textbf{f)}\\
				- Kollisionserkennung würde 2. Antenne ben\"otigen, die während des Sendens empf\"angt 
				\noindent\;\;\;$\Rightarrow$ schwieriger, teurer \\
				-  Hidden-Terminal-Problem wahrscheinlicher\\\\
			\textbf{g)}\\
				- Nachteil: ineffizient (weil die Daten mitunter nur immer mal wieder auf den Kanälen
				 gesendet werden)\\
				- Vorteil: keine Kollisionen//Echtzeitfaehig\\\\
			\textbf{h)}\\
				- ICMP erkennt zwar einige Fehlerzust\"ande, macht aber IP zu keinem zuverl\"assigen 
				Protokoll\\
			\textbf{i)}\\
				- Es geht nur um die Weiterleitung von lokalem in globales Netz, daher nur Quelladresse
				 (IP-Header) und -port (TCP-Header). Auszerdem m\"ussen die beiden Checksummen neu
				  berechnet werden, daher 4 Punkte.\\
	\section{Transportschicht}
		\subsection{Schiebefensterprotokoll}	
			\textbf{a)}\\
				- Ja, denn bei Go-Back-N w\"urde keine ACK3 gesendet werden, bevor nicht eine ACK2 
				gesendet wurde.\\
			\textbf{b)}\\
				- A sendet Pakete, kann also erst das Fenster schieben wenn ACKs ankommen. B empf\"angt,
				 verschiebt also sein Fenster wenn Pakete ankommen.
				\begin{align*}
					&A: ||\:00\;|\;01\;|\;10\;|\;11 \\
					&B: ||\;10\;|\;11\;|\;00\;|\;01
				\end{align*}\\
			\textbf{b)}\\
				- A sendet Paket \textbf{00-11}, aber das ACK 00 von B geht verloren. B hat sein Fenster
				auf die	n\"achsten Pakete \textbf{00-11} verschoben. A sendet das alte Paket \textbf{00}, 
				doch B denkt es sei das neue \textbf{00}.\\
		\subsection{Durchsatz:}
			\textbf{a)}
				\begin{align*}			 
					S &= \frac{1}{1+2a}(1-p)\\ 
					&= \frac{1}{1+2(RD/L}(1-p)\\
					&= \frac{1}{11}(1-p) = \frac{0,97}{11} = 8,8%
				\end{align*}
			\textbf{b)}
				\begin{align*}
				2a = 11 > W \Rightarrow S 	&= \frac{W(1-p)}{(1-p+Wp)(1+2ap)}\\
											&= \frac{0,97 * 4}{(0.97+4*0.03)(1+0,97*10)}\\ 
											&= 0,32 (VL 03_98, warning ongoing discussion)
				\end{align*}
		\subsection{CRC:}
			\textbf{a)}		
		\begin{align*}
		    11100&0110000 : 10111 \\
    		10111&			\\
      		 1011&0110000	\\
      		 1011&1			\\
                 &1110000 	\\
                 &10111		\\
                 &\;\,101100	\\
                 &\;\,10111	\\
                 &\;\;\;\;\;\;\;\;\;10
		\end{align*}
		Ergebnis der "schriftlichen Division an die Nachricht anhaengen. (Pr\"ufdaten sind also \textbf{10}.\\\\
			\textbf{b)}\\
				Die Fehler sind nicht zusammenh\"angend aber ungerade und k\"onnen daher erkannt werden da
				G den Faktor x+1 enthält. (Burstfehler k\"onnen auch nicht nicht geflippte Bits enthalten)
		\subsection{TCP-Leistungsanalyse}
			\textbf{siehe Uebung 5.1}
	
	\begin{small}
		Diese Sammlung bassiert auf dem RK L\"osungspad des faui2k13 (\url{https://pad.stuve.fau.de/p/RK}) und enthält möglicherweise Fehler!
	\end{small}			
\end{document}
