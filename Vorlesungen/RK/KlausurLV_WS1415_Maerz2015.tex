%RK Klausur Loesungsversuch (faui2k13)
\documentclass{article}
\usepackage{amsmath}
%\usepackage{epsf}
\DeclareMathSizes{10}{10}{10}{10}
\setlength{\parindent}{0pt}
\title{Klausur M\"arz 2015}
\begin{document}
	\maketitle
	\section{Allgemeine Fragen}
		\subsection{FTP}
			\textbf{a)}\\
				- Verschiedene TCP-Verbinndungen f\"ur Daten\"ubertragung und Streuerung/Befehle\\
				- Steuerung w\"ahrend Daten\"ubertragen m\"oglich\\
			\textbf{b)}\\
				- im Aktive Mode initialisiert der Server die Datenverbinndung zum Client nach Anfrage\\
				- im Passive Mode tun dies der Client selbst, das hat den Vorteil, dass die Firewall des Clients dann wahrscheinlich nicht die Verbinndung des Servers blockt\\
		\subsection{Adressierung}
			\textbf{a)}\\
				- ARP \\
				- BROTKAST an alle Ger\"ate mit der IP des gesuchten Ger\"ats als Payload\\
				- gesuchtes Ger\"at antwortet mit eigener MAC\\
				- Schicht 2 (Sicherungsschicht)\\
			\textbf{b)}\\
				- Nein, ARP ist bereits ein Gegenbeispiel\\
			\textbf{c)}\\
				Ja, da beide im gleichen Subnetz sind
				\begin{align*}
					&HostB\;00001010\;00000000\;000000|01\;00000100 \\
					&HostC\;00001010\;00000000\;000000|11\;00001000
				\end{align*}
				Teilnehmer: $(2^10)$-2 = 1022
			\textbf{d)}\\
				- Vorteil: Auch bitfehler im IP-Header k\"onnen erkannt werden\\
				- Nachteil: Verletzung des Schichtenprinzips
		\subsubsection{Leitungskodierung}
			\textbf{a)}\\
				- selbsttaktend: Empfänger kann Sendertakt aus Signal gewinnen \\
				- gleichstromfrei: kein Gleichanteil im elektrischen Signal \\
			\textbf{b)}\\
				%Postscript include
				%\epsfxsize=10pt
				%\includegraphics[scale=1]{/postscripts/test.eps}
				
				
\end{document}