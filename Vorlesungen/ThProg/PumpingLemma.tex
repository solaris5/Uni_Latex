	\documentclass{article}
	\usepackage{amsmath}
	\usepackage{ amssymb } % >= as \geq
	\usepackage{nccmath}
	\usepackage{upgreek}
	%New Commands
	\newcommand{\xhspace}[0]{\noindent\hspace*{5mm}}
	\DeclareMathSizes{10}{10}{10}{10}
	\setlength{\parindent}{0pt}
	\title{Pumping Lemma f\"ur Regul\"are Sprachen}
	\date{ }
	\begin{document}
	\section*{Pumping Lemma f\"ur Regul\"are Sprachen}
	%\maketitle
	%----------------------------------%
	Konkret geht es in diesem Beispiel um eine Sprache die alle ge\"offneten Klammern auch wieder schlie$\upbeta$en soll, oder allgemeiner, um eine Sprache die sich eine Anzahl merken muss.\\\\	
	\textbf{'L' ist regul\"ar wenn:}
		\[		
			\forall l \geq 1\;.\, \exists w \in L\;mit\; |w|\geq l
		\]
	\textbf{sodass:}\\
		\xhspace	$\forall$ uvz \\
	\textbf{mit:}\\ 
		\xhspace	w=uvz\\
		\xhspace	$|$v$|$ $\geq$ 1\\
		\xhspace	$|$uv$|$ $\leq$ l\\
	\textbf{gilt:}\\
		\xhspace	u$v^{k}$z $\notin$ L \\ \\
	\textbf{Beweis, dass 'L' \underline{NICHT} regul\"ar ist}\\\\
		Sei l $\geq$ 1 gegeben, w\"ahle:
		\[ 
			w\;= 
				\underbrace{(\,(\,(\,...\,(}_{l+k\;mal}
				\hspace{10mm}
				\;a
				\hspace{1cm}
				\underbrace{,\,b\,)\,,b)\,...\,)}_{l+k\;mal}
				\hspace{10mm}
				\in L
		\]
		k als feste Zahl w\"ahlen, z.B. 10:
		\[ 
			w\;= 
				\underbrace{(\,(\,(\,...\,(}_{l+10\;mal}
				\hspace{10mm}
				\;a
				\hspace{1cm}
				\underbrace{,\,b\,)\,,b)\,...\,)}_{l+10\;mal}
				\hspace{10mm}
				\in L
		\]
		\textbf{dann gilt:}\\\\
		\noindent \hspace*{1cm}$\forall$ uvz \hspace{5mm}\textbf{mit}\hspace{5mm}w = uvz
		\\\\
		\noindent \hspace*{1cm}u = $(^k$ \\ \\
		\noindent \hspace*{1cm}v = $(^m$\\ \\
		\noindent \hspace*{1cm}z = $(^{[l+10-k-m]}$
		\hspace{3mm}a\hspace{3mm},b\,)\,,\,b\,)\,...\,,\,b\,)\\\\
		\textbf{dann w\"ahlenn wir:}\\\\
			\noindent \hspace*{1cm}	w' = u$v^0$z = uz $\notin$ L\\\\
		Denn es gehen [k+l+10-k-m = 10+l+m] Klammern auf und nur [l+10] Klammern zu. 
		Da m$\geq$ ist [10+l+m $>$ 10+l].
		D.h. w' ist nicht in 'L' und 'L' damit nicht regul\"ar.
\end{document}